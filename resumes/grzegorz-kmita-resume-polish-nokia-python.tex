

\documentclass[letterpaper,11pt]{article}

\usepackage[LY1]{fontenc}
\usepackage[utf8]{inputenc}
\usepackage{polski}
\usepackage{fourier}   
\usepackage{latexsym}
\usepackage[empty]{fullpage}
\usepackage{titlesec}
\usepackage{marvosym}
\usepackage[usenames,dvipsnames]{color}
\usepackage{verbatim}
\usepackage{enumitem}
\usepackage[hidelinks]{hyperref}
\usepackage{fancyhdr}
\usepackage[english]{babel}
\usepackage{tabularx}
\input{glyphtounicode}

\DeclareTextCompositeCommand{\k}{LY1}{a}
  {\oalign{a\crcr\noalign{\kern-.27ex}\hidewidth\char7}}
\DeclareTextCompositeCommand{\k}{LY1}{e}
  {\oalign{e\crcr\noalign{\kern-.27ex}\hidewidth\char7\hidewidth}}
\DeclareTextCompositeCommand{\k}{LY1}{E}
  {\oalign{E\crcr\hidewidth\char7\hidewidth}}

\DeclareTextCompositeCommand{\'}{LY1}{c}
  {{\ooalign{\hidewidth\raise-.13875ex\hbox{\'{}}\hidewidth\crcr c}}}
\DeclareTextCompositeCommand{\'}{LY1}{s}
  {{\ooalign{\hidewidth\raise-.13875ex\hbox{\'{}}\hidewidth\crcr s}}}
\DeclareTextCompositeCommand{\'}{LY1}{z}
  {{\ooalign{\hidewidth\raise-.13875ex\hbox{\'{}}\hidewidth\crcr z}}}
\DeclareTextCompositeCommand{\'}{LY1}{C}
  {{\ooalign{\hidewidth\raise.65367ex\hbox{\'{}}\hidewidth\crcr C}}}
\DeclareTextCompositeCommand{\'}{LY1}{S}
  {{\ooalign{\hidewidth\raise.65367ex\hbox{\'{}}\hidewidth\crcr S}}}
\DeclareTextCompositeCommand{\'}{LY1}{Z}
  {{\ooalign{\hidewidth\raise.65367ex\hbox{\'{}}\hidewidth\crcr Z}}}

%----------FONT OPTIONS----------
% sans-serif
% \usepackage[sfdefault]{FiraSans}
% \usepackage[sfdefault]{roboto}
% \usepackage[sfdefault]{noto-sans}
% \usepackage[default]{sourcesanspro}

% serif
% \usepackage{CormorantGaramond}
% \usepackage{charter}

\pagestyle{fancy}
\fancyhf{} % clear all header and footer fields
\fancyfoot{}
\renewcommand{\headrulewidth}{0pt}
\renewcommand{\footrulewidth}{0pt}

% Adjust margins
\addtolength{\oddsidemargin}{-0.5in}
\addtolength{\evensidemargin}{-0.5in}
\addtolength{\textwidth}{1in}
\addtolength{\topmargin}{-.5in}
\addtolength{\textheight}{1.0in}

\urlstyle{same}

\raggedbottom
\raggedright
\setlength{\tabcolsep}{0in}

% Sections formatting
\titleformat{\section}{
  \vspace{-4pt}\scshape\raggedright\large
}{}{0em}{}[\color{black}\titlerule \vspace{-5pt}]

% Ensure that generate pdf is machine readable/ATS parsable
\pdfgentounicode=1

%-------------------------
% Custom commands
\newcommand{\resumeItem}[1]{
  \item\small{
    {#1 \vspace{-2pt}}
  }
}

\newcommand{\resumeSubheading}[4]{
  \vspace{-2pt}\item
    \begin{tabular*}{0.97\textwidth}[t]{l@{\extracolsep{\fill}}r}
      \textbf{#1} & #2 \\
      \textit{\small#3} & \textit{\small #4} \\
    \end{tabular*}\vspace{-6pt}
}

\newcommand{\resumeSubSubheading}[2]{
    \item
    \begin{tabular*}{0.97\textwidth}{l@{\extracolsep{\fill}}r}
      \textit{\small#1} & \textit{\small #2} \\
    \end{tabular*}\vspace{-6pt}
}

\newcommand{\resumeProjectHeading}[2]{
    \item
    \begin{tabular*}{0.97\textwidth}{l@{\extracolsep{\fill}}r}
      \small#1 & #2 \\
    \end{tabular*}\vspace{-6pt}
}

\newcommand{\resumeSubItem}[1]{\resumeItem{#1}\vspace{-4pt}}

\renewcommand\labelitemii{$\vcenter{\hbox{\tiny$\bullet$}}$}

\newcommand{\resumeSubHeadingListStart}{\begin{itemize}[leftmargin=0.15in, label={}]}
\newcommand{\resumeSubHeadingListEnd}{\end{itemize}}
\newcommand{\resumeItemListStart}{\begin{itemize}}
\newcommand{\resumeItemListEnd}{\end{itemize}\vspace{-5pt}}

%-------------------------------------------
%%%%%%  RESUME STARTS HERE  %%%%%%%%%%%%%%%%%%%%%%%%%%%%

\begin{document}

%----------HEADING----------
% \begin{tabular*}{\textwidth}{l@{\extracolsep{\fill}}r}
%   \textbf{\href{http://sourabhbajaj.com/}{\Large Sourabh Bajaj}} & Email : \href{mailto:sourabh@sourabhbajaj.com}{sourabh@sourabhbajaj.com}\\
%   \href{http://sourabhbajaj.com/}{http://www.sourabhbajaj.com} & Mobile : +1-123-456-7890 \\
% \end{tabular*}

\begin{center}
    \textbf{\Huge \scshape Grzegorz Kmita} \\ \vspace{1pt}
    \small +48 576 566 560 $|$ \href{mailto:grzegorzkmita@tuta.io}{\underline{grzegorzkmita@tuta.io}} $|$
    \href{https://linkedin.com/in/grzegorzkmita}{\underline{linkedin.com/in/grzegorzkmita}} $|$
    \href{https://github.com/jirafey}{\underline{github.com/jirafey}}
\end{center}

%-----------EDUKACJA-----------
\section{Edukacja}
  \resumeSubHeadingListStart
    \resumeSubheading
      {Zachodniopomorski Uniwersytet Technologiczny}{Szczecin, Polska} 
      {Inżynier informatyki}{Paź 2022 -- (Planowo) Lip 2026}
  \resumeSubHeadingListEnd
%-----------EXPERIENCE-----------
\section{Doświadczenie}
  \resumeSubHeadingListStart

    \resumeSubheading
      {Summer Trainee - Python Developer}{Lip 2023 -- Wrz 2023}
      {Nokia}{Wrocław, Polska}
      % \resumeItemListStart
        % \resumeItem{}
        % \resumeItem{}
        % \resumeItem{}
      % \resumeItemListEnd

    \resumeSubheading
      {Working Student - Python Developer}{Paź 2023 -- Paź 2024}
      {Nokia}{Wrocław, Polska}
      % \resumeItemListStart
        % \resumeItem{}
        % \resumeItem{}
        % \resumeItem{}
      % \resumeItemListEnd
  \resumeSubHeadingListEnd

%-----------PROJEKTY-----------
\section{Projekty}
    \resumeSubHeadingListStart
    %grzegorzkmita.com%
        \resumeProjectHeading
          {\textbf{\href{https://grzegorzkmita.com}
    {\underline{grzegorzkmita.com}}}  $|$ {\href{https://github.com/jirafey/grzegorzkmita.com}
    {\underline{GitHub}}} $|$ \emph{HTML, CSS, JavaScript, Git, Vercel}}{2020 -- Teraz}
          \resumeItemListStart
            \resumeItem{Utworzyłem personalną stronę typu portfolio z forku open-source projektu z polirytmicznymi animacjami jako tło dla strony}
            \resumeItem{Zintegrowałem Vercela z GitHubem. Po każdym \emph{git pushu} dochodzi do aktualizacji strony, jeżeli przejdzie on wszystkie kontrole od Vercela.}
          \resumeItemListEnd

    %           \resumeProjectHeading
    %              {\textbf {Program mnożący macierze z pliku} $|$ {\href{https://github.com/jirafey/pong}
    % {\underline{GitHub}}} $|$ \emph{C, Git}}{Sty 2023}
    %       \resumeItemListStart
    %     \resumeItem{Program stworzyłem używając wskaźników i dynamicznej alokacji pamięci.}
    %     \resumeItem{Zaimplementowałem wpisywanie macierzy do pliku, generowanie macierzy losowych, wypisywanie macierzy losowej do pliku txt.}
    %   \resumeItemListEnd
      \resumeProjectHeading
            %Square Madness%
          {\textbf {\href{https:jirafey.itch.io/square-madness}
    {\underline{Square Madness}}} $|$ {\href{https://github.com/jirafey/square-madness}
    {\underline{GitHub}}}  $|$ \emph{Python, pygame, Git}}{Cze 2022}
          \resumeItemListStart
            \resumeItem{\emph{Jednoosobowy space shooter 2D. Wrogowie podążają za graczem próbując go unicestwić, a gracz unika pocisków nieprzyjaciół i ulepsza swój statek.}}
            \resumeItem{Stworzyłem w pełni funkcjonalne menu gry, łącznie z ustawieniami audio i video (pełny ekran/okno).}
            \resumeItem{Dodałem zakończenie, podpowiedzi dla gracza, udoskonaliłem m.in. system otrzymywania obrażeń, UI i dostosowałem poziom trudności}
            \resumeItem{Skomponowałem muzykę, utworzyłem grafikę koncepcyjną oraz dodałem SFX.}
          \resumeItemListEnd
      \resumeProjectHeading
            %Teacup Adventure%
                {\textbf {\href{https:jirafey.itch.io/teacup-adventure}
    {\underline{Teacup Adventure}}} $|$ {\href{https://github.com/jirafey/teacup-adventure}
    {\underline{GitHub}}} $|$ \emph{Python, pygame, Git}}{Lut 2022}
      \resumeItemListStart
        \resumeItem{\emph{Jednoosobowa platformówka 2D z interesującą mechaniką - gdy użytkownik skacze, z postaci wylatuje kropla, którą należy podnieść by ponownie skoczyć.}}
        \resumeItem{Skomponowałem muzykę, dodałem SFX, utworzyłem grafikę użytą w grze.}

        \resumeItem{Naprawiłem system kolizji i dostosowałem poziom trudności.}
      \resumeItemListEnd

    %   %Pong%
    %                 {\textbf {\href{https:jirafey.itch.io/pong}
    % {\underline{Pong}}} $|$ {\href{https://github.com/jirafey/pong}
    % {\underline{GitHub}}}  $|$ \emph{Python, turtle, Git}}{Lut 2022}
    %       \resumeItemListStart
    %     \resumeItem{\emph{Dwuoosobowa lokalna gra 2D (Retro Pong) - 2 graczy za pomocą swoich paletek odbijają piłeczkę, należy przebić piłeczkę w taki sposób by przeciwnikowi nie udało się jej odbić.}}
    %     \resumeItem{Zaprogramowałem m.in. poruszanie się, system kolizji, combo i system punktów.}
        
    %   \resumeItemListEnd


    \resumeSubHeadingListEnd

%
%-----------UMIEJĘTNOŚCI PROGRAMISTYCZNE-----------
\section{Umiejętności programistyczne}
\begin{itemize}[leftmargin=0.15in, label={}]
    \small{\item{
     \textbf{Języki}{: Python, C/C++, JavaScript, Bash, Matlab, Arduino, HTML/CSS} \\
     \textbf{Technologie}{: Linux, Git, GitHub, GitLab CI/CD, Jira, Vercel, VS Code, PyCharm, CLion, NeoVim, Unity, Access, Excel} \\
     \textbf{Biblioteki}{: paramiko, gspread(Google Sheets API), PyQt5, pygame}
    }}
\end{itemize}
\section{Języki obce}

\begin{itemize}[leftmargin=0.15in, label={}]
    \small{\item{
    \textbf{Polski}{: ojczysty} \\
        \textbf{Angielski}{: C1} \\
            \textbf{Chiński(mandaryński)}{: B1} \\
                \textbf{Niemiecki}{: A2} \\
                    \textbf{Niderlandzki}{: A1} \\

    }}
\end{itemize}
    
%-------------------------------------------
\end{document}

